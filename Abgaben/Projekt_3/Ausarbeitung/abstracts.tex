\documentclass[a4paper, 12pt]{report}
\usepackage[utf8]{inputenc}
\usepackage[T1]{fontenc}
\usepackage{textcomp}
\usepackage[ngerman]{babel}
\usepackage{graphicx}
\usepackage{geometry}
\usepackage{csquotes}
\usepackage[toc,page]{appendix}
\usepackage{titlesec}
\usepackage{listings}
\usepackage{float}
\usepackage[hang,flushmargin]{footmisc} 
\usepackage{makecell}
\usepackage{amsmath}% http://ctan.org/pkg/amsmath

\usepackage[default]{cantarell} %% Use option "defaultsans" to use cantarell as sans serif only
\usepackage{hyperref}
\usepackage{helvet}
\usepackage[eulergreek]{sansmath}
\usepackage{amsfonts}
\usepackage{movie15}
\usepackage{url}
\usepackage{lscape}
\usepackage{changepage}
\usepackage[usenames,dvipsnames]{color}
\usepackage{matlab-prettifier}
\lstset{style=Matlab-editor}

\usepackage{tikz}
\usetikzlibrary{calc,positioning,shapes,decorations.pathreplacing}
\usetikzlibrary{fit,fadings,shadows,patterns,math}
\usetikzlibrary{shapes.arrows,shapes.geometric}
\usetikzlibrary{arrows,arrows.meta,decorations.markings}
\usetikzlibrary{decorations.pathmorphing}

\graphicspath{ {./images/} }
\geometry{margin=1.25in}

\newcommand*{\shifttext}[2]{%
  \settowidth{\@tempdima}{#2}%
  \makebox[\@tempdima]{\hspace*{#1}#2}%
}
\newcommand{\fakesection}[1]{%
  \par\refstepcounter{section}% Increase section counter
  \sectionmark{#1}% Add section mark (header)
  \addcontentsline{toc}{section}{\protect\numberline{\thesection}#1}% Add section to ToC
  % Add more content here, if needed.
}
\titleformat{\chapter}[display]
  {\normalfont\bfseries}{}{0pt}{\Large}
\titleformat*{\section}{\large\bfseries}

\renewcommand{\footnoterule}{%
  \kern -3pt
  \hrule width \textwidth height 1pt
  \kern 2pt
}

\newcommand\blfootnote[1]{%
  \begingroup
  \renewcommand\thefootnote{}\footnote{#1}%
  \addtocounter{footnote}{-1}%
  \endgroup
}

% This is the color used for MATLAB comments below
\definecolor{MyDarkGreen}{rgb}{0.0,0.4,0.0}

% For faster processing, load Matlab syntax for listings
\lstloadlanguages{Matlab}%
\lstset{language=Matlab,                        % Use MATLAB
        frame=single,                           % Single frame around code
        basicstyle=\small\ttfamily,             % Use small true type font
        keywordstyle=[1]\color{Blue}\bfseries,        % MATLAB functions bold and blue
        keywordstyle=[2]\color{Purple},         % MATLAB function arguments purple
        keywordstyle=[3]\color{Blue}\underbar,  % User functions underlined and blue
        identifierstyle=,                       % Nothing special about identifiers
                                                % Comments small dark green courier
        commentstyle=\usefont{T1}{pcr}{m}{sl}\color{MyDarkGreen}\small,
        stringstyle=\color{Purple},             % Strings are purple
        showstringspaces=false,                 % Don't put marks in string spaces
        tabsize=5,                              % 5 spaces per tab
        %
        %%% Put standard MATLAB functions not included in the default
        %%% language here
        morekeywords={xlim,ylim,var,alpha,factorial,poissrnd,normpdf,normcdf, quadprog, fmincon, fminunc},
        %
        %%% Put MATLAB function parameters here
        morekeywords=[2]{on, off, interp},
        %
        %%% Put user defined functions here
        morekeywords=[3]{confunNeqG},
        %
        morecomment=[l][\color{Blue}]{...},     % Line continuation (...) like blue comment
        inputencoding=latin1
        }
        \lstset{literate=%
        {Ö}{{\"O}}1
        {Ä}{{\"A}}1
        {Ü}{{\"U}}1
        {ß}{{\ss}}1
        {ü}{{\"u}}1
        {ä}{{\"a}}1
        {ö}{{\"o}}1
      }

\begin{document}

\begin{center}
    \vspace*{2em}
    \normalsize 3. Projekt\\
    \vspace*{1em}
    \normalsize \textbf{\textit{Penalty-Verfahren \& SQP-Verfahren}}\\
    \vspace*{4em}
    \normalsize im Fach\\
    \vspace*{1em}
    \large Numerische Optimierung\\
    \vspace*{30em}
    \normalsize Juni 2020\\
    \vspace*{1em}
    \normalsize Maximilian Gaul
\end{center}

\thispagestyle{empty}

\newpage

\subsubsection{Aufgabe 1}
Für das Problem

$$\text{min } f(x) = 1000 - x_1^2 - 2x_2^2 - x_3^2 - x_1x_2 - x_1x_3$$

unter den Nebenbedingungen

$$h_1(x) = x_1^2 + x_2^2 + x_3^2 - 25 = 0$$
$$h_2(x) = 8x_1 + 14x_2 - 7x_3 - 56 = 0$$

kann man die Lagrange-Funktion aufstellen

$$L(x, \mu) = 1000 - x_1^2 - 2x_2^2 - x_3^2 - x_1x_2 - x_1x_3 + \mu_1 (x_1^2 + x_2^2 + x_3^2 - 25) + \mu_2 (8x_1 + 14x_2 - 7x_3 - 56)$$

sowie deren Ableitung bestimmen

$$\nabla L = \nabla f(x) + \mu_1\nabla h_1(x) + \mu_2 \nabla h_2(x)$$

mit

$$\nabla f(x) = \begin{bmatrix}-2x_1 - x_2 - x_3\\-4x_2 - x_1\\-2x_3-x_1\end{bmatrix} \text{  } \nabla h_1(x) = \begin{bmatrix}2_x1\\2x_2\\2x_3\end{bmatrix} \text{  } \nabla h_2(x) = \begin{bmatrix}8\\14\\7\end{bmatrix}$$

Daraus ergibt sich die Stationaritätsgleichung

$$\begin{bmatrix}-2x_1 - x_2 - x_3\\-4x_2 - x_1\\-2x_3-x_1\end{bmatrix} + \mu_1 \begin{bmatrix}2x_1\\2x_2\\2x_3\end{bmatrix} + \mu_2 \begin{bmatrix}8\\14\\7\end{bmatrix} = \begin{bmatrix}0\\0\\0\end{bmatrix}$$

Zudem müssen folgende Zulässigkeiten gelten: $h_1(\hat{x}) = 0, h_2(\hat{x}) = 0$.

Außerdem muss \textit{LICQ} gelten: $\nabla h_1(\hat{x}), \nabla h_2(\hat{x})$ sind linear unabhängig.

\subsubsection{Aufgabe 2}
Implementierung siehe \lstinline[basicstyle=\ttfamily\color{black}]|BFGS_Pen.m| und
\lstinline[basicstyle=\ttfamily\color{black}]|ArmijoPen.m| sowie \lstinline[basicstyle=\ttfamily\color{black}]|Projekt_3.m|.

Das Verfahren \textit{Armijo-Regel mit Aufweitung} überprüft zuerst, ob die Armijo-Regel aus dem allgemeinen Verfahren mit der als
Parameter übergebenen Startschrittweite erfüllt ist. Falls nicht wird die aktuelle Schrittweite solange mit einem Faktor $\beta \in (0, 1)$
multipliziert, bis die Armijo-Regel erfüllt ist. Sollte dagegen die Armijo-Regel von Anfang an schon erfüllt sein, wird versucht die
Schrittweite so lange zu vergrößern, bis die Armijo-Regel gerade nicht mehr erfüllt ist. Dadurch verhindert man zu kleine Schrittweiten
weswegen das normale Armijo-Verfahren auch nicht effizient ist (mit Aufweitung hingegen ist es effizient). Dadurch, dass die als
Parameter übergebene Schrittweite im BFGS-Verfahren als neue Schrittweite auch für den nächsten Funktionsaufruf von \textit{Armijo-Pen}
verwendet wird, versucht man schon eine bereits gut passende Schrittweite für den nächsten Schritt zu bekommen.\par
Im globalen BFGS-Verfahren wird eine Kombination aus verschiedenen Abbruchbedingung gewählt. Zuerst eine Grenze an den Betrag des Gradienten,
wodurch die Genauigkeit des Ergebnisses festgelegt wird. Außerdem wird eine obere Grenze für die Anzahl der Durchläufe festgelegt.\par
Anschließend wird überprüft ob mit der Approximation der Hesse-Matrix eine gültige Abstiegsrichtung vorliegt, falls nicht, wird der
negative Gradient verwendet (anschließend hätte man noch die Matrix B z.B. auf $I$ zurücksetzen können). Mit der Abstiegsrichtung
wird dann auch die Schrittweite mit dem oben beschriebenen Armijo-Verfahren mit Aufweitung berechnet. Mit Schrittweite und Richtung
kann der Ergebnisvektor $x$ nun geupdated werden. Zuletzt wird für den nächsten Schritt eine Update-Formel, die auf zweimaliger Anwendung
von Sherman-Morrison-Woodbury basiert, auf die Hesse-Matrix-Approximation angewendet.\par
Sowohl das BFGS-Verfahren und das Armijo-Verfahren mit Aufweitung erwarten eine Funktion, die Funktionswert und Gradient liefert.

\subsubsection{Aufgabe 3}
Tests siehe \lstinline[basicstyle=\ttfamily\color{black}]|Projekt_3.m|.

Zum Minimieren wird dem BFGS-Verfahren folgende Funktion übergeben:

$$P(x, r) = f(x) + \frac{1}{2}r\left((h_1(x))^2 + (h_2(x))^2\right)$$
$$P(x,r) = 1000 - x_1^2 - 2x_2^2 - x_3^2 - x_1x_2 - x_1x_3 + \frac{1}{2} r \left( \left(x_1^2 + x_2^2 + x_3^2 - 25\right)^2 + \left(8x_1 + 14x_2 + 7x_3 - 56\right)^2 \right)$$

Als Gradient erhält die BFGS-Funktion

$$\nabla P(x, r) = \nabla f(x) + r\left(h_1(x)\cdot \nabla h_1(x) + h_2(x)\cdot \nabla h_2(x)\right)$$
$$\nabla P(x, r) = \begin{bmatrix}-2x_1 - x_2 - x_3\\-4x_2-x_1\\-2x_3-x_1\end{bmatrix} + r\left(h_1(x) \cdot \begin{bmatrix}2x_1\\2x_2\\2x_3\end{bmatrix} + h_2(x)\cdot \begin{bmatrix}8\\14\\7\end{bmatrix}\right)$$

\begin{figure}[H]
  \centering
  \def\arraystretch{1.25}
  \begin{tabular}{l|c|r}
    \hline
    \textbf{Schritt} & \textbf{$x$} & \textbf{$P(x)$}\\
    \hline
    1 & $[25.00, 35.00, 45.00]^T$ & $3.9308\cdot 10^{10}$\\
    10 & $[-10.18, -13.95, -15.62]^T$ & $0.1157\cdot 10^{10}$\\
    25 & $[3.16, 4.81, 4.06]^T$ & $1.2063 \cdot 10^7$\\
    100 & $[1.53, 0.95, 4.43]^T$ & $1.3411\cdot 10^4$\\
    500 & $[2.75, 0.35, 4.16]^T$ & $962.49$\\
    961 & $[3.51, 0.22, 3.55]^T$ & $961.72$\\
    \hline
  \end{tabular}
  \caption{Verlauf von \lstinline[basicstyle=\ttfamily\color{black}]|BFGS_Pen| mit $r = 5000$ und Genauigkeit $10^{-6}$}
\end{figure}

Mit Matlab

\begin{figure}[H]
  \centering
  \def\arraystretch{1.25}
  \begin{tabular}{l|c|r}
    \hline
    \textbf{Schritt} & \textbf{$x$} & \textbf{$P(x)$}\\
    \hline
    1 & $[25.00, 35.00, 45.00]^T$ & $3.9308\cdot 10^{10}$\\
    3 & k.A. & $1.5288\cdot 10^9$\\
    5 & k.A. & $1.5364\cdot 10^8$\\
    7 & k.A. & $9.2605\cdot 10^6$\\
    10 & k.A. & $35476$\\
    19 & $[1.29, 0.89, 4.75]^T$ & $968.92$\\
    \hline
  \end{tabular}
  \caption{Verlauf von \lstinline[basicstyle=\ttfamily\color{black}]|fminunc| mit $r = 5000$ und\\
  \lstinline[basicstyle=\ttfamily\color{black}]|OptimalityTolerance = 1e-6|}
\end{figure}

Für \lstinline[basicstyle=\ttfamily\color{black}]|fmincon| benötigt man eine extra Matlab-Funktion:

\begin{lstlisting}
% Gleichheitsbedingungen aus Aufgabe 2 für fmincon
function [c,ceq] = confuneqF(h1, h2, x)
    % Nonlinear inequality constraints
    c = [];
    % Nonlinear equality constraints
    ceq = [h1(x), h2(x)];
end\end{lstlisting}

um die Funktion letztendlich so aufzurufen:\par
\lstinline|fmincon(f,x0,[],[],[],[],[],[], confuneqF, options)|

Zusammengefasst:

\begin{figure}[H]
  \centering
  \def\arraystretch{1.25}
  \begin{tabular}{l|c|c|c|r}
    \hline
    \textbf{Funktion} & \textbf{Endergebnis} & \textbf{Genauigkeit} & \textbf{Anzahl Schritte} & \textbf{$f(x)$}\\
    \hline
    \lstinline|BFGS| & $[3.51, 0.22, 3.55]^T$ & $10^{-6}$ & 961 & $961.72$\\
    \lstinline|fminunc| & $[1.29, 0.89, 4.75]^T$ & $10^{-6}$ & 19 & $966.92$ \\
    \lstinline|fmincon| & $[0.33, 4.68, -1.73]$ & $10^{-6}$ & 10 &$952.14$\\
    \hline
  \end{tabular}
  \caption{Ergebnisse für alle betrachteten Funktionen}
\end{figure}

\subsubsection{Aufgabe 4}

Das Problem

$$f(x) = \text{min } x_1$$

unter den Nebenbedingungen

$$g_1(x) = x_1^2 + x_2^2 - 1 \leq 0$$
$$g_2(x) = x_1 + x_2 - \gamma \leq 0 \text{, } \gamma \geq -\sqrt{2}$$

lässt sich so graphisch darstellen:

\begin{figure}[H]
  \centering
  \begin{tikzpicture}
    \newcommand*{\from}{-3}
    \newcommand*{\too}{3}
    \newcommand*{\drawFrom}{-1}
    \newcommand*{\drawTo}{1}
    \newcommand*{\scaleFactor}{1}

    \tikzstyle{interval border}=[thin, red];
    \draw[-{Latex}] (\from*\scaleFactor,0) -- (\too*\scaleFactor,0) node[right] {$x_1$};
    \draw[-{Latex}] (0,\from*\scaleFactor) -- (0,\too*\scaleFactor) node[above] {$x_2$};

    \draw (0, 0) node[xshift=-0.21cm, yshift=-0.25cm, font=\small] {$0$};

    \foreach \x in {-3, -2, -1, 1, 2, 3} {
      \draw (\x*\scaleFactor, 0) node[yshift=-0.25cm, font=\small] {$\x$};
      \draw (0,\x*\scaleFactor) node[xshift=-0.425cm, font=\small] {$\x$};
    }

    \draw[red, fill=orange, opacity=0.5, thick] (0,0) circle (1 cm);

    \draw[blue, thick] (1.6, -3.014) -- (-1.6, 0.18579);
    \draw[blue, thick] (-1.6, 3.014) -- (1.6, -0.18579);

    \node[xshift=-2cm, yshift=0.5cm, font=\tiny, draw, thin, rectangle] (ppre) at (-1.6, 0.18579) {$x_1 + x_2 \leq -\sqrt{2}$};
    \draw[-{Latex}] (ppre.east) -- +(+5pt,0pt) -- (-1.6, 0.18579);

    \node[xshift=-2cm, yshift=0.5cm, font=\tiny, draw, thin, rectangle] (ppre) at (-1.6, 3.014) {$x_1 + x_2 \leq \sqrt{2}$};
    \draw[-{Latex}] (ppre.east) -- +(+5pt,0pt) -- (-1.6, 3.014);

    \node[xshift=2cm, yshift=1.5cm, font=\tiny, draw, thin, rectangle] (ppre) at (0.9, 0.4359) {$x_1^2 + x_2^2 \leq 1$};
    \draw[-{Latex}] (ppre.west) -- +(-5pt,0pt) -- (0.9, 0.4359);

    \draw[black, fill=black, thick] (0,0) circle (0.025cm);
    \draw[-{Latex}] (0, 0) -- node[font=\tiny, above=-0.1cm] {$\nabla f$} (1,0);

    \fill[opacity=0.125, blue] (1.6, -3.014) -- (-1.6, 0.18579) -- (-1.6, 3.014) -- (1.6, -0.18579) -- cycle;

  \end{tikzpicture}
  \caption{Graphische Darstellung der Zielfunktion und Nebenbedingungen als Scheibe und Fläche, gültige Punkte müssen in der Schnittmenge
    aus Blau und Orange liegen}
\end{figure}

Anhand der Lagrange-Funktion:

$$L(x, \lambda) = x_1 + \lambda_1(x_1^2 + x_2^2 - 1) + \lambda_2(x_1 + x_2 - \gamma)$$

und den Ableitungen:

$$\nabla f(x) = \begin{bmatrix}1 \\ 0\end{bmatrix} \text{, } \nabla g_1(x) = \begin{bmatrix}2x_1\\2x_2\end{bmatrix} \text{, } \nabla g_2(x) = \begin{bmatrix}1\\1\end{bmatrix}$$

kann man die KKT-Bedingungen aufstellen:

$$\nabla_x L(x, \lambda) = \nabla f(x) + \lambda_1\nabla g_1(x) + \lambda_2\nabla g_2(x) = 0 \text{ (Stationarität)}$$

$$ \Leftrightarrow \begin{bmatrix}1\\0\end{bmatrix} + \lambda_1\cdot \begin{bmatrix}2x_1\\2x_2\end{bmatrix} + \lambda_2\cdot \begin{bmatrix}1\\1\end{bmatrix} = \begin{bmatrix}0\\0\end{bmatrix}$$

$$\lambda_1\text{, } \lambda_2 \geq 0\text{, } \lambda_1\cdot (x_1^2 + x_2^2 - 1) = 0 \text{, } \lambda_2\cdot (x_1 + x_2 - \gamma) = 0 \text{ (Komplementarität)}$$
$$g_1(x) \leq 0 \text{, } g_2(x) \leq 0 \text{ (Zulässigkeit)}$$

Um die Komplementarität zu erfüllen kann man nun verschiedene Faktoren gleich Null setzen.\par

Für $\lambda_1 = 0$ und $\lambda_2 = 0$ erhält man einen Widerspruch in der Stationarität, ebenso für
$\lambda_1 = 0$ und $\lambda_2 \neq 0$.\par
Für $\lambda_1 \neq 0$ und $\lambda_2 = 0$ erhält man:

$$\begin{bmatrix}1\\0\end{bmatrix} + \lambda_1 \cdot \begin{bmatrix}2x_1\\2x_2\end{bmatrix} = \begin{bmatrix}0\\0\end{bmatrix}$$

mit

$$1 + 2\lambda_1x_1 = 0$$
$$2\lambda_1x_2 = 0 \Rightarrow^{\lambda_1 \neq 0} x_2 = 0$$

Da $\lambda_1 \neq 0$ muss wegen der Komplementarität zwangsläufig gelten\\
$x_1^2 + x_2^2 = 1 \Leftrightarrow x_1 = \pm \sqrt{1 - x_2^2}$ bzw. $x_1 = \pm 1$. Nun einsetzen

$$1 + 2\lambda_1 = 0 \Leftrightarrow \lambda_1 = -\frac{1}{2} \text{  (ungültig) }$$
$$1 - 2\lambda_1 = 0 \Leftrightarrow \lambda_1 = \frac{1}{2}$$

Damit erhält man den KKT-Punkt: $x_1 = -1 \text{ , } x_2 = 0$.\par

Für $\lambda_1 \neq 0$ und $\lambda_2 \neq 0$ erhält man

$$x_1 + x_2 - \gamma = 0 \Leftrightarrow x_1 = \gamma - x_2$$

$$x_1^2 + x_2^2 - 1 = 0 \Leftrightarrow (\gamma - x_2)^2 + x_2^2 - 1 = 0 \Leftrightarrow x_2 = \pm \sqrt{\frac{2(1-\gamma^2) + \gamma^2}{4}} + \frac{1}{2}\gamma$$

$$\Leftrightarrow x_2 = \pm \frac{1}{2}\cdot \sqrt{\gamma^2 + 2(1-\gamma^2)} + \frac{1}{2}\gamma \Rightarrow x_1 = \gamma - \left(\pm \frac{1}{2}\cdot \sqrt{\gamma^2 + 2(1-\gamma^2)} + \frac{1}{2}\gamma \right)$$

Für $x_1 = \gamma - \frac{1}{2}\sqrt{\gamma^2 + 2(1 - \gamma^2)} - \frac{1}{2}\gamma \Leftrightarrow x_1 = \frac{1}{2}\gamma - \frac{1}{2}\sqrt{\gamma^2 + 2(1 - \gamma^2)}$ und\\
$x_2 = \frac{1}{2}\sqrt{\gamma^2 - 2(1 - \gamma^2)} + \frac{1}{2}\gamma$ erhält man die Stationaritätsgleichung

$$\begin{bmatrix}1\\0\end{bmatrix} + \lambda_1 \cdot \begin{bmatrix}2\cdot\left(\frac{1}{2}\gamma - \frac{1}{2}\sqrt{\gamma^2 + 2(1 - \gamma^2)}\right)\\2\cdot\left(\frac{1}{2}\sqrt{\gamma^2 + 2(1 - \gamma^2)} + \frac{1}{2}\gamma\right)\end{bmatrix} + \lambda_2\cdot\begin{bmatrix}1\\1\end{bmatrix} = \begin{bmatrix}0\\0\end{bmatrix}$$

Mit

$$\lambda_1 \cdot \left(\sqrt{\gamma^2 + 2(1 - \gamma^2)} + \gamma\right) + \lambda_2 = 0 \Leftrightarrow \lambda_2 = -\lambda_1 \cdot \left(\sqrt{\gamma^2 + 2(1 - \gamma^2)} + \gamma\right)$$

kann man einsetzen in die 1. Gleichung:

$$1 + \lambda_1\cdot \left(\gamma - \sqrt{\gamma^2 + 2(1 - \gamma^2)}\right) - \lambda_1\cdot \left(\sqrt{\gamma^2 + 2(1 - \gamma^2)} + \gamma\right) = 0$$

und erhält

$$\lambda_1 = \frac{1}{2\sqrt{\gamma^2 + 2(1 - \gamma^2)}} \text{ , } \lambda_2 = - \frac{\sqrt{\gamma^2 + 2(1 - \gamma^2)} + \gamma}{2\sqrt{\gamma^2 + 2(1 - \gamma^2)}}$$

Für $\gamma \neq \pm \sqrt{2}$ ist $\lambda_1$ immer positiv. Für $-\sqrt{2} < \gamma \leq -1$ ist $\lambda_2$ ebenfalls positiv.
D.h. $x_1 = \frac{1}{2}\gamma - \frac{1}{2}\sqrt{\gamma^2 + 2(1 - \gamma^2)}$ und $x_2 = \frac{1}{2}\sqrt{\gamma^2 - 2(1 - \gamma^2)} + \frac{1}{2}\gamma$ ist ein KKT-Punkt.

Weiterhin erhält man für $x_1 = \gamma - \left(-\frac{1}{2}\sqrt{\gamma^2 + 2(1 - \gamma^2)} + \frac{1}{2}\gamma\right)$\\
$\Leftrightarrow x_1 = \frac{1}{2}\gamma + \frac{1}{2}\sqrt{\gamma^2 + 2(1 - \gamma^2)}$ und $x_2 = - \frac{1}{2}\sqrt{\gamma^2 + 2(1 - \gamma^2)} + \frac{1}{2}\gamma$

$$\begin{bmatrix}1\\0\end{bmatrix} + \lambda_1 \cdot \begin{bmatrix}2\cdot \left(\frac{1}{2}\gamma + \frac{1}{2}\sqrt{\gamma^2 + 2(1 - \gamma^2)}\right)\\2\cdot \left(-\frac{1}{2}\sqrt{\gamma^2 + 2(1-\gamma^2)} + \frac{1}{2}\gamma\right)\end{bmatrix} + \lambda_2 \cdot \begin{bmatrix}1\\1\end{bmatrix} =\begin{bmatrix}0\\0\end{bmatrix}$$

Und

$$\lambda_1 = - \frac{1}{2\sqrt{\gamma^2 + 2(1 - \gamma^2)}} \text{ , } \lambda_2 = \frac{-\sqrt{\gamma^2 + 2(1 - \gamma^2)} + \gamma}{2\sqrt{\gamma^2 + 2(1 - \gamma^2)}}$$

Da $\lambda_1$ für $\gamma \neq \pm \sqrt{2}$ immer negativ ist, ist dies kein KKT-Punkt.

\subsubsection{Aufgabe 5}

\begin{figure}[H]
  \centering
  \begin{tikzpicture}
    \newcommand*{\from}{-3}
    \newcommand*{\too}{3}
    \newcommand*{\drawFrom}{-1}
    \newcommand*{\drawTo}{1}
    \newcommand*{\scaleFactor}{1}

    \tikzstyle{interval border}=[thin, red];
    \draw[-{Latex}] (\from*\scaleFactor,0) -- (\too*\scaleFactor,0) node[right] {$x_1$};
    \draw[-{Latex}] (0,\from*\scaleFactor) -- (0,\too*\scaleFactor) node[above] {$x_2$};

    \draw (0, 0) node[xshift=-0.21cm, yshift=-0.25cm, font=\small] {$0$};

    \foreach \x in {-3, -2, -1, 1, 2, 3} {
      \draw (\x*\scaleFactor, 0) node[yshift=-0.25cm, font=\small] {$\x$};
      \draw (0,\x*\scaleFactor) node[xshift=-0.425cm, font=\small] {$\x$};
    }

    \draw[red, thick] (0,0) circle (1 cm);

    \draw[blue, thick] (1.6, -3.014) -- (-1.6, 0.18579);
    \draw[blue, thick] (-1.6, 3.014) -- (1.6, -0.18579);
    \draw[blue, thick] (-1.6, 1.6) -- (1.6, -1.6);

    \node[xshift=-2cm, yshift=0.5cm, font=\tiny, draw, thin, rectangle] (ppre) at (-1.6, 0.18579) {$x_1 + x_2 = -\sqrt{2}$};
    \draw[-{Latex}] (ppre.east) -- +(+5pt,0pt) -- (-1.6, 0.18579);

    \node[xshift=-2cm, yshift=0.5cm, font=\tiny, draw, thin, rectangle] (ppre) at (-1.6, 1.6) {$x_1 + x_2 = 0$};
    \draw[-{Latex}] (ppre.east) -- +(+5pt,0pt) -- (-1.6, 1.6);

    \node[xshift=-2cm, yshift=0.5cm, font=\tiny, draw, thin, rectangle] (ppre) at (-1.6, 3.014) {$x_1 + x_2 = \sqrt{2}$};
    \draw[-{Latex}] (ppre.east) -- +(+5pt,0pt) -- (-1.6, 3.014);

    \node[xshift=2cm, yshift=1.5cm, font=\tiny, draw, thin, rectangle] (ppre) at (0.9, 0.4359) {$x_1^2 + x_2^2 = 1$};
    \draw[-{Latex}] (ppre.west) -- +(-5pt,0pt) -- (0.9, 0.4359);

    \draw[black, fill=black, thick] (0,0) circle (0.025cm);
    \draw[-{Latex}] (0, 0) -- node[font=\tiny, above=-0.1cm] {$\nabla f$} (1,0);

  \end{tikzpicture}
  \caption{Graphische Darstellung der Zielfunktion und Nebenbedingungen als Kreis und Gerade, gültige Punkte liegen auf der Schnittmenge
    zwischen oranger und den blauen Linien}
\end{figure}

Anhand der Lagrange-Funktion:

$$L(x,\mu) = x_1 + \mu_1\left( x_1^2 + x_2^2 - 1 \right) + \mu_2\left( x_1 + x_2 - \gamma \right)$$

und den Ableitungen

$$\nabla f(x) = \begin{bmatrix}1 \\ 0\end{bmatrix} \text{, } \nabla h_1(x) = \begin{bmatrix}2x_1\\2x_2\end{bmatrix} \text{, } h_2(x) = \begin{bmatrix}1\\1\end{bmatrix}$$

kann man die KKT-Bedingungen aufstellen:

$$\begin{bmatrix}1 \\ 0\end{bmatrix} + \mu_1 \cdot \begin{bmatrix}2x_1\\2x_2\end{bmatrix} + \mu_2 \cdot \begin{bmatrix}1\\1\end{bmatrix} = \begin{bmatrix}0\\0\end{bmatrix} \text{ (Stationarität)}$$

sowie

$$h_1(x) = 0 \text{, } h_2(x) = 0 \text{ (Zulässigkeit)}$$

Mit

$$x_1 = \frac{-\mu_2 - 1}{2\mu_1} \text{ und } x_2 = \frac{-\mu_2}{2\mu_1}$$

kann man in die 2. Nebenbedingung einsetzen

$$ \frac{-\mu_2 - 1}{2\mu_1} + \frac{-\mu_2}{2\mu_1} = \gamma$$

und erhält

$$\mu_2 = -\gamma\mu_1 - \frac{1}{2}$$

Was man wiederum in die 1. Nebenbedingung einsetzen kann

$$ \left( \frac{-\left(-\gamma\mu_1 - \frac{1}{2}\right) - 1}{2\mu_1} \right)^2 + \left( \frac{-\left(-\gamma\mu_1 - \frac{1}{2}\right)}{2\mu_1} \right)^2 = 1$$

Dann erhält man

$$\mu_1 = \pm \frac{1}{2\cdot \sqrt{2 - \gamma^2}} \text{, } \mu_2 = \mp \frac{\gamma}{2\cdot \sqrt{2 - \gamma^2}} - \frac{1}{2}$$

Insgesamt erhält man also die KKT-Punkte:

$$x_1 = \sqrt{2 - \gamma^2} \cdot \left( \frac{\gamma}{2\cdot\sqrt{2 - \gamma^2}} - \frac{1}{2} \right) \text{ , } x_2 = \sqrt{2 - \gamma^2} \cdot \left( \frac{\gamma}{2\cdot\sqrt{2 - \gamma^2}} + \frac{1}{2} \right)$$

und

$$x_1 = -\sqrt{2 - \gamma^2}\cdot \left(-\frac{\gamma}{2\cdot \sqrt{2 - \gamma^2}} - \frac{1}{2}\right) \text{ , } x_2 = -\sqrt{2 - \gamma^2} \cdot \left( \frac{1}{2} - \frac{\gamma}{2\cdot\sqrt{2 - \gamma^2}} \right)$$

\subsubsection{Aufgabe 6}

Man löst das Problem

$$\text{min } x_1$$

mit den Nebenbedingungen

$$g_1(x) = x_1^2 + x_2^2 \leq 1$$
$$g_2(x) = x_1 + x_2 \leq 1$$

mit dem Matlab-Befehl \lstinline[basicstyle=\ttfamily\color{black}]|fmincon| indem man die Ungleichungsnebenbedingungen in einer separaten Funktion
definiert:

\begin{lstlisting}
  % Ungleichheitsbedingungen aus Aufgabe 4 für fmincon
  function [c,ceq] = confunNeqG(g1, g2, x)
      % Nonlinear inequality constraints
      c = [g1(x), g2(x)];
      % Nonlinear equality constraints
      ceq = [];
  end\end{lstlisting}

und dann \lstinline|fmincon(g, x0, [], [], [], [], [], [], confunNeqG)| aufruft. Das Ergebnis ist $x_1 = -1, x_2 = 0$.

Ähnlich verhält es sich mit dem Problem

$$\text{min } x_1$$

und den Nebenbedingungen

$$h_1(x) = x_1^2 + x_2^2 = 1$$
$$h_2(x) = x_1 + x_2 = 1$$

Hier wird folgende Matlab-Funktion verwendet:

\begin{lstlisting}
  % Gleichheitsbedingungen aus Aufgabe 5 für fmincon
  function [c,ceq] = confuneqG(h1, h2, x)
      % Nonlinear inequality constraints
      c = [];
      % Nonlinear equality constraints
      ceq = [h1(x), h2(x)];
  end\end{lstlisting}

Das Ergebnis ist $x_1 = 0, x_2 = 1$.

Konkrete Implementierung siehe \lstinline[basicstyle=\ttfamily\color{black}]|Projekt_3.m|.

\subsubsection{Aufgabe 7}

Man bringt die Nebenbedingungen zuerst in die Form:

$$g_1(x) = -1 + x_1 + x_2 \leq 0$$
$$g_2(x) = -1 + x_1 - x_2 \leq 0$$
$$g_3(x) = -1 - x_1 + x_2 \leq 0$$
$$g_4(x) = -1 - x_1 - x_2 \leq 0$$

Die Stationaritätsgleichung lässt sich dann wie folgt aufstellen:

$$\nabla  f(x) + \lambda_1\nabla g_1(x) + \lambda_2\nabla g_2(x) + \lambda_3\nabla g_3(x) + \lambda_4\nabla g_4(x) = 0$$

$$\Leftrightarrow$$

$$\begin{bmatrix}2(x_1 - 1.5)\\4(x_2 - t)^3\end{bmatrix} + \lambda_1 \begin{bmatrix}1\\1\end{bmatrix} + \lambda_2 \begin{bmatrix}1\\-1\end{bmatrix} + \lambda_3 \begin{bmatrix}-1\\1\end{bmatrix} + \lambda_4 \begin{bmatrix}-1\\-1\end{bmatrix} = \begin{bmatrix}0\\0\end{bmatrix}$$

Ebenso die Komplementaritätsbedingungen:

$$\lambda_1\cdot g_1(x) = 0$$
$$\lambda_2\cdot g_2(x) = 0$$
$$\lambda_3\cdot g_3(x) = 0$$
$$\lambda_4\cdot g_4(x) = 0$$
$$\lambda_1 \text{ , } \lambda_2 \text{ , } \lambda_3 \text{ , } \lambda_4 \geq 0$$

Wenn man den Punkt $\hat{x}$ in die Nebenbedingungen einsetzt sieht man, welche aktiv sind:

$$g_1(\hat{x}) = 0$$
$$g_2(\hat{x}) = 0$$
$$g_3(\hat{x}) = -2$$
$$g_4(\hat{x}) = -2$$

Um die Komplementaritätsbedingungen zu erfüllen, muss nun gelten:\\
$\lambda_3 = \lambda_4 = 0$. Die Stationaritätsgleichung verkürzt sich daher zu

$$\begin{bmatrix}-1\\-4t^3\end{bmatrix} + \lambda_1 \begin{bmatrix}1\\1\end{bmatrix} + \lambda_2\begin{bmatrix}1\\-1\end{bmatrix} = \begin{bmatrix}0\\0\end{bmatrix}$$

$\nabla g_1$ und $\nabla g_2$ sind linear unabhängig (\textit{LICQ} erfüllt). Man erhält dann

$$\lambda_1 = -\lambda_2 + 1$$

und

$$-4t^3 + \lambda_1 - \lambda_2 = 0 \Leftrightarrow \lambda_2 = - \frac{4t^3 - 1}{2}$$

Daher

$$\lambda_1 = \frac{4t^3 - 1}{2} + 1$$

Um die Komplementaritätsbedingungen zu erfüllen muss gelten

$$\frac{4t^3 - 1}{2} + 1 \geq 0 \land - \frac{4t^3 - 1}{2} \geq 0$$

Für $t \leq \frac{1}{\sqrt[3]{2^2}}$ und $t \geq -\frac{1}{\sqrt[3]{2^2}}$ (ungefähr $t \in [-0.63, 0.63]$) ist $\hat{x}$ ein KKT-Punkt.

\subsubsection{Aufgabe 8}

Wenn man die Nebenbedingungen in ein Koordinatensystem zeichnet:

\begin{figure}[H]
  \centering
  \begin{tikzpicture}
    \newcommand*{\from}{-3}
    \newcommand*{\too}{3}
    \newcommand*{\scaleFactor}{1}

    \tikzstyle{interval border}=[thin, red];
    \draw[-{Latex}] (\from*\scaleFactor,0) -- (\too*\scaleFactor,0) node[right] {$x_1$};
    \draw[-{Latex}] (0,\from*\scaleFactor) -- (0,\too*\scaleFactor) node[above] {$x_2$};

    \draw (0, 0) node[xshift=-0.21cm, yshift=-0.25cm, font=\small] {$0$};

    \foreach \x in {-3, -2, -1, 1, 2, 3} {
      \draw (\x*\scaleFactor, 0) node[yshift=-0.25cm, font=\small] {$\x$};
      \draw (0,\x*\scaleFactor) node[xshift=-0.425cm, font=\small] {$\x$};
    }

    \draw[thick, blue] (-2, 3) node[above=0.0cm, font=\tiny] {(1)} -- (2, -1);
    \fill[opacity=0.125, blue] (-2, 3) -- (2, -1) --(2, -2) -- (-2, 2) -- cycle;

    \draw[thick, red] (-2, -3) -- (2, 1) node[right=0.0cm, font=\tiny] {(2)};
    \fill[opacity=0.125, red] (-2, -3) -- (2, 1) -- (2, 2) -- (-2, -2) -- cycle;

    \draw[thick, green] (-2, -1) -- (2, 3) node[right=0.0cm, font=\tiny] {(3)};
    \fill[opacity=0.125, green] (-2, -1) -- (2, 3) -- (2, 2) -- (-2, -2) -- cycle;

    \draw[thick, purple] (-2, 1) node[below=0.0cm, font=\tiny] {(4)} -- (2, -3);
    \fill[opacity=0.125, purple] (-2, 1) -- (2, -3) -- (2, -2) -- (-2, 2) -- cycle;

    \draw[thick, black, fill=black] (1, 0) circle (0.075cm);
    \draw[thick, black, fill=black] (0, 1) circle (0.075cm);
    \draw[thick, black, fill=black] (-1, 0) circle (0.075cm);
    \draw[thick, black, fill=black] (0, -1) circle (0.075cm);

  \end{tikzpicture}
  \caption{Nebenbedingungen von $f(x_1, x_2)$}
\end{figure}

erkennt man, das nur einseitig beschränkte Flächen entstehen, wenn entweder

$$ \text{(1) und (3)}$$
$$ \text{(1) und (2)}$$
$$ \text{(2) und (4)}$$
$$ \text{(3) und (4)}$$

oder nur eine NB gleichzeitig aktiv sind / ist (bzw. nur diese Kombinationen haben linear unabhängige Gradienten nach \textit{LICQ}).

Ausgehend von der Stationaritätsgleichung:

$$ \begin{bmatrix}2(x_1 - 1.5)\\4(x_2 - 1)^3\end{bmatrix} + \lambda_1 \begin{bmatrix}1\\1\end{bmatrix} + \lambda_2 \begin{bmatrix}1\\-1\end{bmatrix} + \lambda_3 \begin{bmatrix}-1\\1\end{bmatrix} + \lambda_4 \begin{bmatrix}-1\\-1\end{bmatrix} = \begin{bmatrix}0\\0\end{bmatrix}$$

probiert man nun diese Möglichkeiten durch d.h. setzt Nebenbedingungen aktiv und formt durch die Komplementaritätsbedingungen um.

\begin{itemize}
  \item $\lambda_1 = \lambda_2 = 0, \lambda_3 \neq 0, \lambda_4 \neq 0$\\
  Aus den Nebenbedingungen (3) und (4) erhält man:\\
  $$ -x_1 + x_2 = 1 \Leftrightarrow x_2 = 1 + x_1$$
  $$ -x_1 - x_2 = 1 \Leftrightarrow x_1 = -1 \Rightarrow x_2 = 0$$

  Einsetzen in die Stationaritätsgleichung:
  $$\begin{bmatrix}-5\\-4\end{bmatrix} + \lambda_3 \begin{bmatrix}-1\\1\end{bmatrix} + \lambda_3 \begin{bmatrix}-1\\-1\end{bmatrix} = \begin{bmatrix}0\\0\end{bmatrix}$$
  und man erhält $\lambda_3 = -0.5 \text{ , } \lambda_4 = -4.5$. Das ist also keine gültige Lösung.

  \item $\lambda_1 = \lambda_3 = 0, \lambda_2 \neq 0, \lambda_4 \neq 0$\\
  Man erhält aus den NB (2) und (4): $x_1 = 0 \text{ , } x_2 = -1$ und nach Einsetzen:

  $$\begin{bmatrix}-3\\-32\end{bmatrix} + \lambda_2 \begin{bmatrix}1\\-1\end{bmatrix} + \lambda_4 \begin{bmatrix}-1\\-1\end{bmatrix} = \begin{bmatrix}0\\0\end{bmatrix}$$

  Daraus dann: $\lambda_2 = -14.5 \text{ , } \lambda_4 = -17.5$, auch keine gültige Lösung.

  \item $\lambda_2 = \lambda_4 = 0, \lambda_1 \neq 0, \lambda_3 \neq 0$\\
  Aus den NB (1) und (3): $x_1 = 0, x_2 = 1$ und

  $$\begin{bmatrix}-3\\0\end{bmatrix} + \lambda_1 \begin{bmatrix}1\\1\end{bmatrix} + \lambda_3 \begin{bmatrix}-1\\1\end{bmatrix} = \begin{bmatrix}0\\0\end{bmatrix}$$
  dann: $\lambda_1 = \frac{3}{2}, \lambda_2 = -\frac{3}{2}$, auch keine gültige Lösung.

  \item $\lambda_3 = \lambda_4 = 0, \lambda_1 \neq 0, \lambda_2 \neq 0$\\
  Aus NB (1) und (2) erhält man $x_1 = 1, x_2 = 0$ und nach Einsetzen:
  $$\begin{bmatrix}-1\\-4\end{bmatrix} + \lambda_1 \begin{bmatrix}1\\1\end{bmatrix} + \lambda_1 \begin{bmatrix}1\\-1\end{bmatrix} = \begin{bmatrix}0\\0\end{bmatrix}$$
  auch $\lambda_1 = \frac{5}{2}, \lambda_2 = -\frac{3}{2}$, also auch keine gültige Lösung.
\end{itemize}

Bisher wurde kein KKT-Punkt gefunden. Nun werden noch alle Nebenbedingungen einzeln getestet:

\begin{itemize}
  \item $\lambda_1 = \lambda_2 = \lambda_3 = 0, \lambda_4 \neq 0$\\
  Aus NB (4) erhält man $x_1 = -1 - x_2$, Einsetzen in die Stationaritätsgleichung liefert
  $$\begin{bmatrix}2(-1-x_2)\\4(x_2-1)^3\end{bmatrix} + \lambda_4 \begin{bmatrix}-1\\-1\end{bmatrix} = \begin{bmatrix}0\\0\end{bmatrix}$$
  Die 1. Gleichung umformen zu $\lambda_4 = -2 - 2x_2$ und einsetzen in die\\
  2. Gleichung: $4(x_2 - 1)^3 + 2 + 2x_2 = 0 \Leftrightarrow 4x_2^3 - 12x_2^2 + 14x_2 - 2 = 0$\\
  $\Leftrightarrow x_2 \cong 0.16488 \Rightarrow x_1 \cong -1.16488 \Rightarrow \lambda_4 \cong -2.32976$, keine gültige Lösung

  \item $\lambda_1 = \lambda_2 = \lambda_4 = 0, \lambda_3 \neq 0$\\
  Mit NB (3) erhält man $x_2 = 1 + x_1$, wiederum Einsetzen:
  $$\begin{bmatrix}2(x_1 - 1.5)\\4(x_2 -1)^3\end{bmatrix} +\lambda_3 \begin{bmatrix}-1\\1\end{bmatrix} = \begin{bmatrix}0\\0\end{bmatrix}$$
  Umformen: $\lambda_3 = 2x_1 - 3$ und in die 2. Gleichung einsetzen:\\
  $4(1 + x_1 - 1)^3 + 2x_1 - 3 = 0 \Leftrightarrow x_1 \cong 0.72808 \Rightarrow x_2 \cong 1.72808$\\
  $\lambda_3 \cong -1.54384$, keine Lösung

  \item $\lambda_1 = \lambda_3 = \lambda_4 = 0, \lambda_2 \neq 0$\\
  Mit NB (2) erhält man $x_1 = 1 + x_2$, einsetzen:
  $$\begin{bmatrix}2(x_1 - 1.5)\\4(x_2 -1)^3\end{bmatrix} + \lambda_2 \begin{bmatrix}1\\-1\end{bmatrix} = \begin{bmatrix}0\\0\end{bmatrix}$$
  1. Gleichung umformen zu $\lambda_2 = 1 - 2x_2$ und in 2. Gleichung einsetzen:\\
  $4(x_2 - 1)^3 -1 + 2x_2 = 0 \Leftrightarrow x_2 \cong 0.61454 \Rightarrow x_1 = 1.61454$\\
  $\lambda_2 = -0.22908$, keine Lösung

  \item $\lambda_2 = \lambda_3 = \lambda_4 = 0, \lambda_1 \neq 0$\\
  Mit NB(1) erhält man $x_1 = 1 - x_2$, einsetzen in Stationaritätsgleichung
  $$\begin{bmatrix}2(x_1 - 1.5)\\4(x_2 -1)^3\end{bmatrix} + \lambda_1 \begin{bmatrix}1\\1\end{bmatrix} = \begin{bmatrix}0\\0\end{bmatrix}$$
  Die 1. Gleichung umformen: $\lambda_1 = 1 + 2x_2$, einsetzen in die 2. Gleichung:\\
  $4(x_2 - 1)^3 + 2x_2 + 1 = 0 \Leftrightarrow x_2 \cong 0.27192 \Rightarrow x_1 \cong 0.72808$\\
  $\lambda_1 = 1.54384$. Hier also das erste Ergebnis mit $\lambda \geq 0$.

\end{itemize}

Die Lösung des Systems ist demnach $x_1 \cong 0.72808, x_2 \cong 0.27192$\\
mit $f(x_1, x_2) \cong 0.87687$

\subsubsection{Aufgabe 9}
Betrachtet wird das Problem

$$\text{min } g(x) = (x_1 - 1.5)^2 + (x_2 - 0.75)^4$$

unter den Nebenbedingungen

$$g_1(x) = x_1 + x_1 \leq 1$$
$$g_2(x) = x_1 - x_1 \leq 1$$
$$g_3(x) = -x_1 + x_1 \leq 1$$
$$g_4(x) = -x_1 - x_1 \leq 1$$

mit der Lagrange-Funktion

$$L(x) = (x_1 - 1.5)^2 + (x_2 - 0.75)^4$$
$$+\lambda_1(-1 + x_1 + x_2)$$
$$+\lambda_2(-1 + x_1 - x_2)$$
$$+\lambda_3(-1-x_1+x_2)$$
$$+\lambda_4(-1-x_1-x_2)$$

Für das SQP-Verfahren benötigt man sowohl den Gradienten der Zielfunktion als auch den der Lagrange-Funktion:

$$\nabla g(x) = \begin{bmatrix}2(x_1 - 1.5)\\4(x_2 - 0.75)^3\end{bmatrix} \text{  } \nabla L = \begin{bmatrix}2(x_1 - 1.5) + \lambda_1 + \lambda_2 - \lambda_3 - \lambda_4\\4(x_2 - 0.75)^3 + \lambda_1 - \lambda_2 + \lambda_3 - \lambda_4\end{bmatrix}$$

und weiterhin auch die Hesse-Matrix der Lagrange-Funktion

$$\nabla^2L = \begin{bmatrix}2 & 0\\0 & 12(x_2 - 0.75)^2\end{bmatrix}$$

Weiterhin benötigt man die Gradienten der Nebenbedingungen:

$$\nabla g_1 = \begin{bmatrix}1\\1\end{bmatrix} \text{  } \nabla g_2 = \begin{bmatrix}1\\-1\end{bmatrix} \text{  } \nabla g_3 = \begin{bmatrix}-1\\1\end{bmatrix} \text{  } \nabla g_4 = \begin{bmatrix}-1\\-1\end{bmatrix}$$

Mit $B = \nabla^2L$ kann man das quadratische Hilfsproblem aufstellen:

$$\text{min }\nabla g(x)^Td + \frac{1}{2}d^TBd$$

unter den Nebenbedingungen

$$g_1(x) + \nabla g_1(x)^Td \leq 0 \Leftrightarrow \nabla g_1(x)^Td \leq -g_1(x)$$
$$g_2(x) + \nabla g_2(x)^Td \leq 0 \Leftrightarrow \nabla g_2(x)^Td \leq -g_2(x)$$
$$g_3(x) + \nabla g_3(x)^Td \leq 0 \Leftrightarrow \nabla g_3(x)^Td \leq -g_3(x)$$
$$g_4(x) + \nabla g_4(x)^Td \leq 0 \Leftrightarrow \nabla g_4(x)^Td \leq -g_4(x)$$

In Matrixschreibweise $A\cdot d = b$

$$\begin{bmatrix}1 & 1\\1 & -1\\-1 & 1\\-1 & -1\end{bmatrix}^T\cdot d \leq -\begin{bmatrix}g_1(x)\\g_2(x)\\g_3(x)\\g_4(x)\end{bmatrix}$$

\newpage

Das Problem wird mit \lstinline[basicstyle=\ttfamily\color{black}]|quadprog| wie folgt gelöst
(konkrete Implementierung \lstinline[basicstyle=\ttfamily\color{black}]|Projekt_3.m|)

\begin{lstlisting}
  g1 = @(x) -1 + x(1) + x(2);
  g2 = @(x) -1 + x(1) - x(2);
  g3 = @(x) -1 - x(1) + x(2);
  g4 = @(x) -1 - x(1) - x(2);

  g_deriv = @(x) [2 * ( x(1) - 1.5 ); 4 * ( x(2) - 0.75).^3];
  g_hessian = @(x) [ 2, 0; 0, 12 * ( x(2) - 0.75).^2 ];
  
  x0 = [0; 0];
  n = x0;
  
  b = -[g1(n), g2(n), g3(n), g4(n)];
  A = [1, 1; 1, -1; -1, 1; -1, -1];

  ret = quadprog(g_hessian(n), g_deriv(n), A, b);
  
  n = n + ret;\end{lstlisting}

Nach dem 1. Schritt erhält man $n = \begin{bmatrix}0.9214\\0.0786\end{bmatrix}$ mit $g(n) = 0.5380$.

\subsubsection{Aufgabe 10}

Für die Ungleichungsnebenbedingungen benötigt man eine weitere Funktion in Matlab:

\begin{lstlisting}
  % Ungleichheitsbedingungen aus Aufgabe 8 für fmincon
  function [c,ceq] = confunNeqG(g1, g2, g3, g4, x)
      % Nonlinear inequality constraints
      c = [g1(x), g2(x), g3(x), g4(x)];
      % Nonlinear equality constraints
      ceq = [];
  end\end{lstlisting}

So dass man das Problem mit Startwert $x^0 = \begin{bmatrix}2\\3\end{bmatrix}$ wie folgt lösen kann
(konkrete Implementierung siehe \lstinline[basicstyle=\ttfamily\color{black}]|Projekt_3.m|)

\begin{lstlisting}
  t = 0.75;
  g = @(x) ( x(1) - 1.5 ).^2 + ( x(2) - t ).^4;

  options = optimoptions("fmincon", "Algorithm", "sqp");
  ret = fmincon(g, [2; 3], [], [], [], [], [], [], confunNeqG, options);\end{lstlisting}

Man erhält $x = \begin{bmatrix}0.9141\\0.0859\end{bmatrix}$ mit $g(x)=0.5378$.

\end{document}