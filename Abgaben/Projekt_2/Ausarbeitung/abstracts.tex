\documentclass[a4paper, 12pt]{report}
\usepackage{graphicx}
\usepackage[utf8]{inputenc}
\usepackage[ngerman]{babel}
\usepackage{geometry}
\usepackage{csquotes}
\usepackage[toc,page]{appendix}
\usepackage{titlesec}
\usepackage{listings}
\usepackage{float}
\usepackage[hang,flushmargin]{footmisc} 
\usepackage{makecell}
\usepackage{amsmath}% http://ctan.org/pkg/amsmath
\usepackage[utf8]{inputenc}
\usepackage[default]{cantarell} %% Use option "defaultsans" to use cantarell as sans serif only
\usepackage[T1]{fontenc}
\usepackage{hyperref}
\usepackage{helvet}
\usepackage[eulergreek]{sansmath}
\usepackage{amsfonts}
\usepackage{movie15}
\usepackage{url}
\usepackage{lscape}

\usepackage{tikz}
\usetikzlibrary{calc,positioning,shapes,decorations.pathreplacing}
\usetikzlibrary{fit,fadings,shadows,patterns,math}
\usetikzlibrary{shapes.arrows,shapes.geometric}
\usetikzlibrary{arrows,arrows.meta,decorations.markings}
\usetikzlibrary{decorations.pathmorphing}

\usepackage{pgfplots}

\graphicspath{ {./images/} }
\geometry{margin=1.25in}

\newcommand*{\shifttext}[2]{%
  \settowidth{\@tempdima}{#2}%
  \makebox[\@tempdima]{\hspace*{#1}#2}%
}
\newcommand{\fakesection}[1]{%
  \par\refstepcounter{section}% Increase section counter
  \sectionmark{#1}% Add section mark (header)
  \addcontentsline{toc}{section}{\protect\numberline{\thesection}#1}% Add section to ToC
  % Add more content here, if needed.
}
\titleformat{\chapter}[display]
  {\normalfont\bfseries}{}{0pt}{\Large}
\titleformat*{\section}{\large\bfseries}

\renewcommand{\footnoterule}{%
  \kern -3pt
  \hrule width \textwidth height 1pt
  \kern 2pt
}

\newcommand\blfootnote[1]{%
  \begingroup
  \renewcommand\thefootnote{}\footnote{#1}%
  \addtocounter{footnote}{-1}%
  \endgroup
}

\begin{document}

\begin{center}
    \vspace*{2em}
    \normalsize 2. Projekt\\
    \vspace*{1em}
    \normalsize \textbf{\textit{Quasi-Newton-Verfahren \& Gauß-Newton-Verfahren}}\\
    \vspace*{4em}
    \normalsize im Fach\\
    \vspace*{1em}
    \large Numerische Optimierung\\
    \vspace*{30em}
    \normalsize Juni 2020\\
    \vspace*{1em}
    \normalsize Maximilian Gaul
\end{center}

\thispagestyle{empty}

\newpage

\subsubsection{Aufgabe 1}
Siehe Programmcode in \lstinline[basicstyle=\ttfamily\color{black}]|Project2.m|.

\newpage

\newgeometry{margin=0.25in}
\begin{landscape}
\subsubsection{Aufgabe 2}
$$\scriptscriptstyle I - \frac{s(As)^T}{s^TAs} + \frac{A^{-1}yy^T}{y^Ts} + \frac{(s - A^{-1}y)s^TA + s(s - A^{-1}y)^TA}{y^Ts} - \frac{(s - A^{-1}y)s^TAs(As)^T + s(s - A^{-1}y)^TAs(As)^T}{y^Tss^TAs} + \frac{(s - A^{-1}y)s^Tyy^T + s(s - A^{-1}y)^Tyy^T}{(y^Ts)^2} - \frac{(s - A^{-1}y)^Tyss^TA}{(y^Ts)^2} + \frac{(s - A^{-1}y)^Tyss^TAs(As)^T}{(y^Ts)^2s^TAs} - \frac{(s - A^{-1}y)^Tyss^Tyy^T}{(y^Ts)^2y^Ts}$$
$$\scriptscriptstyle I - \frac{ss^TA}{s^TAs} + \frac{A^{-1}yy^T}{y^Ts} + \frac{ss^TA - A^{-1}ys^TA + s(s^T - (A^{-1}y)^T)A}{y^Ts} - \frac{ss^TAss^TA - A^{-1}ys^TAss^TA + s(s^T - (A^{-1}y)^T)Ass^TA}{y^Tss^TAs} + \frac{ss^Tyy^T - A^{-1}ys^Tyy^T + s(s^T - (A^{-1}y)^T)yy^T}{(y^Ts)^2} -\frac{(s^T - (A^{-1}y)^T)yss^TA}{(y^Ts)^2} + \frac{(s^T - (A^{-1}y)^T)yss^TAss^TA}{(y^Ts)^2s^TAs} - \frac{(s^T - (A^{-1}y)^T)yss^Tyy^T}{(y^Ts)^2y^Ts} $$
$$\scriptscriptstyle I - \frac{ss^TA}{s^TAs} + \frac{A^{-1}yy^T}{y^Ts} + \frac{2ss^TA - A^{-1}ys^TA - sy^T}{y^Ts} - \frac{2ss^TA - A^{-1}ys^TA - sy^T}{y^Tss^TAs}\cdot ss^TA + \frac{2ss^T - A^{-1}ys^T - sy^T(A^{-1})^T}{(y^Ts)^2}\cdot yy^T -\frac{s^Ty - y^T(A^{-1})^T}{(y^Ts)^2}\cdot ss^TA + \frac{(s^T - y^T(A^{-1})^T)yss^TAss^TA}{(y^Ts)^2s^TAs} - \frac{(s^T - y^T(A^{-1})^T)yss^Tyy^T}{(y^Ts)^2y^Ts} $$
\end{landscape}
\restoregeometry

\newpage

\subsubsection{Aufgabe 3}
Wenn die Suchrichtung des BFGS Verfahrens:

$$d = -B\cdot \nabla f(x) $$

keine Abstiegsrichtung ist, d.h. die Bedingung:

$$\nabla f(x)^T \cdot d < 0 $$

nicht erfüllt ist, muss das Verfahren 'resettet' werden. In diesem Fall bietet es sich an, die Suchrichtung auf den
negativen Gradienten zu setzen:

$$d = -\nabla f(x)$$

Da nun die Abstiegsrichtung nicht mehr zur approximierten Inversen der Hesse-Matrix $B$ passt, muss diese ebenfalls neu
bestimmt werden. Hierzu bieten sich verschiedene Möglichkeiten an:

\begin{itemize}
  \item Man könnte wie beim Start des BFGS-Verfahrens $B = I$ setzen
\end{itemize}

\end{document}