\documentclass[a4paper, 12pt]{report}
\usepackage{graphicx}
\usepackage[utf8]{inputenc}
\usepackage[ngerman]{babel}
\usepackage{geometry}
\usepackage{csquotes}
\usepackage[toc,page]{appendix}
\usepackage{titlesec}
\usepackage{listings}
\usepackage{float}
\usepackage[hang,flushmargin]{footmisc} 

\usepackage{amsmath}% http://ctan.org/pkg/amsmath
\usepackage[utf8]{inputenc}
\usepackage[default]{cantarell} %% Use option "defaultsans" to use cantarell as sans serif only
\usepackage[T1]{fontenc}

\usepackage{helvet}
\usepackage[eulergreek]{sansmath}

\usepackage{tikz}
\usetikzlibrary{shapes.arrows}
\usetikzlibrary{positioning, shapes}
\usepackage{pgfplots}

\graphicspath{ {./images/} }
\geometry{margin=1.25in}

\newcommand*{\shifttext}[2]{%
  \settowidth{\@tempdima}{#2}%
  \makebox[\@tempdima]{\hspace*{#1}#2}%
}
\newcommand{\fakesection}[1]{%
  \par\refstepcounter{section}% Increase section counter
  \sectionmark{#1}% Add section mark (header)
  \addcontentsline{toc}{section}{\protect\numberline{\thesection}#1}% Add section to ToC
  % Add more content here, if needed.
}
\titleformat{\chapter}[display]
  {\normalfont\bfseries}{}{0pt}{\Large}
\titleformat*{\section}{\large\bfseries}

\renewcommand{\footnoterule}{%
  \kern -3pt
  \hrule width \textwidth height 1pt
  \kern 2pt
}

\newcommand\blfootnote[1]{%
  \begingroup
  \renewcommand\thefootnote{}\footnote{#1}%
  \addtocounter{footnote}{-1}%
  \endgroup
}

\begin{document}

\begin{center}
    \vspace*{2em}
    \normalsize 1. Projekt\\
    \vspace*{1em}
    \normalsize \textbf{\textit{Ableitungsfreie Methoden}}\\
    \vspace*{4em}
    \normalsize im Fach\\
    \vspace*{1em}
    \large Numerische Optimierung\\
    \vspace*{30em}
    \normalsize Mai 2020\\
    \vspace*{1em}
    \normalsize Maximilian Gaul
\end{center}

\thispagestyle{empty}

\newpage

\subsubsection{Aufgabe 1}
Teilintervalle des Bisektionsverfahrens für das Minimum von $h(x) = e^{-x} + 0.5x^2$ mit dem Startintervall $[0, 1]$:
\\

\subsubsection{Aufgabe 3}
Abbruchkriterien und Strategie für Wahl von $\alpha$...
\\

\subsubsection{Aufgabe 4}
Vergleich Rechenaufwand...
\\

\subsubsection{Aufgabe 5}
Beispiel angeben bei dem Abbruchkriterium ungeeignet ist...
\\

\subsubsection{Aufgabe 6}
Berechnet werden die ersten vier Iterationen des Nelder-Mead-Algorithmus von
$$g(x_1, x_2) = 100\cdot(x_2-2)^4 + (x_1 - 2x_2)^2$$
mit den Parametern $n=2$, $\alpha = \frac{1}{2}$, $\beta = 2$ und $\gamma = 1$.
$$x^{(0,0)} = \begin{bmatrix}4\\2\end{bmatrix}, e_1 = \begin{bmatrix}1\\0\end{bmatrix}, e_2 = \begin{bmatrix}0\\1\end{bmatrix}$$
Damit erhält man die Punkte $x^{(0,1)} = \begin{bmatrix}5\\2\end{bmatrix}, x^{(0,2)} = \begin{bmatrix}4\\3\end{bmatrix}$ und den
Startsimplex
$$S_0 = (\begin{bmatrix}4\\2\end{bmatrix}, \begin{bmatrix}5\\2\end{bmatrix}, \begin{bmatrix}4\\3\end{bmatrix})$$

\textbf{k = 0}\\
$$\text{max}\{f(x^{(0,0)}) = 1600, f(x^{(0,1)}) = 8101, f(x^{(0,2)}) = 1604\} = f(x^{(0,1)})$$
$$s_0 = \frac{1}{2}\cdot (\begin{bmatrix}4\\2\end{bmatrix} + \begin{bmatrix}4\\3\end{bmatrix}) = \begin{bmatrix}4\\\frac{5}{2}\end{bmatrix} \text{und } x_0 = x^{(0,1)} = \begin{bmatrix}5\\2\end{bmatrix}$$
\begin{itemize}
\item Reflexion: $\hat x_0 = \begin{bmatrix}4\\\frac{5}{2}\end{bmatrix} + 1\cdot(\begin{bmatrix}4\\\frac{5}{2}\end{bmatrix} - \begin{bmatrix}5\\2\end{bmatrix}) = \begin{bmatrix}3\\3\end{bmatrix}$ mit $f(\hat x_0) = 25$
\item Expansion: $\hat x_0^* = \begin{bmatrix}4\\\frac{5}{2}\end{bmatrix} + 2\cdot (\begin{bmatrix}3\\3\end{bmatrix} - \begin{bmatrix}4\\\frac{5}{2}\end{bmatrix}) = \begin{bmatrix}2\\\frac{7}{2}\end{bmatrix}$ mit $f(\hat x_0^*) = 25$
\end{itemize}

Nach dem 1. Schritt erhält man den Simplex

$$S_1 = (\begin{bmatrix}4\\2\end{bmatrix}, \begin{bmatrix}2\\\frac{7}{2}\end{bmatrix}, \begin{bmatrix}4\\3\end{bmatrix})$$

\textbf{k = 1}\\
  \subsubsection{Aufgabe 7}
Diskussion: Zuverlässigkeit und Rechenaufwand von \textit{Mutation-Selektion} und \textit{Nelder-Mead}


\end{document}