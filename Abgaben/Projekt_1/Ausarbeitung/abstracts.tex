\documentclass[a4paper, 12pt]{report}
\usepackage{graphicx}
\usepackage[utf8]{inputenc}
\usepackage[ngerman]{babel}
\usepackage{geometry}
\usepackage{csquotes}
\usepackage[toc,page]{appendix}
\usepackage{titlesec}
\usepackage{listings}
\usepackage{float}
\usepackage[hang,flushmargin]{footmisc} 
\usepackage{hyperref}

\usepackage[utf8]{inputenc}
\usepackage[default]{cantarell} %% Use option "defaultsans" to use cantarell as sans serif only
\usepackage[T1]{fontenc}

\usepackage{helvet}
\usepackage[eulergreek]{sansmath}

\usepackage{tikz}
\usetikzlibrary{shapes.arrows}
\usetikzlibrary{positioning, shapes}
\usepackage{pgfplots}

\graphicspath{ {./images/} }
\geometry{margin=1.25in}

\newcommand*{\shifttext}[2]{%
  \settowidth{\@tempdima}{#2}%
  \makebox[\@tempdima]{\hspace*{#1}#2}%
}
\newcommand{\fakesection}[1]{%
  \par\refstepcounter{section}% Increase section counter
  \sectionmark{#1}% Add section mark (header)
  \addcontentsline{toc}{section}{\protect\numberline{\thesection}#1}% Add section to ToC
  % Add more content here, if needed.
}
\titleformat{\chapter}[display]
  {\normalfont\bfseries}{}{0pt}{\Large}
\titleformat*{\section}{\large\bfseries}

\renewcommand{\footnoterule}{%
  \kern -3pt
  \hrule width \textwidth height 1pt
  \kern 2pt
}

\newcommand\blfootnote[1]{%
  \begingroup
  \renewcommand\thefootnote{}\footnote{#1}%
  \addtocounter{footnote}{-1}%
  \endgroup
}

\begin{document}

\begin{center}
    \vspace*{2em}
    \normalsize 1. Projekt\\
    \vspace*{1em}
    \normalsize \textbf{\textit{Ableitungsfreie Methoden}}\\
    \vspace*{4em}
    \normalsize im Fach\\
    \vspace*{1em}
    \large Numerische Optimierung\\
    \vspace*{30em}
    \normalsize Mai 2020\\
    \vspace*{1em}
    \normalsize Maximilian Gaul
\end{center}

\thispagestyle{empty}

\newpage

\textbf{Aufgabe 1}\\
Teilintervalle des Bisektionsverfahrens für das Minimum von $h(x) = e^{-x} + 0.5x^2$ mit dem Startintervall $[0, 1]$:
\\

\textbf{Aufgabe 3}\\
Abbruchkriterien und Strategie für Wahl von $\alpha$...
\\

\textbf{Aufgabe 4}\\
Vergleich Rechenaufwand...
\\

\textbf{Aufgabe 5}\\
Beispiel angeben bei dem Abbruchkriterium ungeeignet ist...
\\

\textbf{Aufgabe 6}\\
Nelder-Mead-Algorithmus von Hand...
\\

\textbf{Aufgabe 7}\\
Diskussion: Zuverlässigkeit und Rechenaufwand von \textit{Mutation-Selektion} und \textit{Nelder-Mead}


\end{document}